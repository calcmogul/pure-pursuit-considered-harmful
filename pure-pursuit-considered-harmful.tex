\documentclass[10pt,conference,compsoc]{IEEEtran}

\usepackage{amsmath}
\DeclareMathOperator{\sgn}{sgn}
\usepackage{bm}
\usepackage{datetime}
\usepackage{enumerate}
\usepackage{graphicx}
\PassOptionsToPackage{hyphens}{url}
  \usepackage[colorlinks=true,linkcolor=blue]{hyperref}
\usepackage{import}
\usepackage{listings}
\usepackage{lmodern}
\usepackage{parskip}
\usepackage{tabulary}

% Load after hyperref
\usepackage[style=super,nonumberlist,toc]{glossaries}
\usepackage[notlof, notlot, nottoc]{tocbibind}
\usepackage{tocloft}

% Set up snippet environment
\newcommand{\listsnippetsname}{\Large List of Snippets}
\newlistof{snippets}{los}{\listsnippetsname}
\cftsetindents{snippets}{\parindent}{1.5\parindent}
\newenvironment{snippet}{
  \renewcommand{\caption}[1]{
    \refstepcounter{snippets}
    \begin{center}
      \par\noindent{\footnotesize Snippet \thesnippets. ##1}
    \end{center}
    \addcontentsline{los}{snippets}
      {\protect \numberline{\thesnippets}{\ignorespaces ##1}}
  }
}
{\par\noindent}

% Disable automatic indent and provide \indent command
\newlength\tindent
\setlength{\tindent}{\parindent}
\setlength{\parindent}{0pt}
\renewcommand{\indent}{\hspace*{\tindent}}

% Paper header
\usepackage{fancyhdr}
\pagestyle{fancy}
\lhead{Pure Pursuit Considered Harmful}
\rhead{\thepage}
\lfoot{}
\rfoot{}
\fancyfoot[C]{}  % remove normal page numbers

\allowdisplaybreaks
\begin{document}

\title{Pure Pursuit Considered Harmful \\ \large Why trajectory trackers should be used instead of pure pursuit}
\author{Tyler Veness}
\maketitle

\section{What is pure pursuit?}

The pure pursuit path tracking algorithm calculates the angular velocity
necessary for a robot moving at constant velocity to reach a point ahead of it
on the path. This lookahead point recedes over time, hence the name \textit{pure
pursuit}.

\section{Theoretical and practical deficiencies}

Pure pursuit is a fun exercise, but it's inferior to other nonlinear
controllers.

I fundamentally disagree with pure pursuit's controller methodology. I think a
trajectory controller should be offered too. Here's why I dislike pure pursuit:
\begin{enumerate}
  \item It doesn't offer direct control over heading (or in the case of swerve,
    the velocity vector).
  \item If the robot is already on the path, it won't stay on it around corners
    due to the lookahead point. This is a fundamental feature of pursuit
    controllers that can't be fixed completely. It's problematic in the presence
    of obstacles.
  \item Because the robot deviates from the path all the time, it's harder to
    meticulously plan where the robot goes.
  \item It has unstable behavior at the ends of paths that has to be worked
    around by stopping the controller early or artificially extending the
    trajectory via the tangent to the last waypoint.
  \item It doesn't let you enforce dynamics constraints on the path, or have
    much control over the time domain response of the robot at all.
\end{enumerate}

Essentially, instead of trajectory and controller tuning, you now have to do
path tuning to make the robot do what you want.

\section{Arguments I've seen for pure pursuit and responses}
\begin{quote}
  There’s a lot of reasons I currently like using it, but they pretty much all
  stem from the ability to do on the fly “generation” instead of generating all
  path states ahead of time.

  So, if you get hit off course by another robot it will be able to react
  gracefully and get back on track instead of over correcting and going max
  speed to get back to what the state should be at that point in time.
\end{quote}

Spline trajectories can be replanned on the fly, even with kinematic and dynamic
constraints.
\begin{quote}
  Another reason this is really useful is that paths don’t need to have a known
  starting state. You can follow a path from a standstill right at the start
  point, far away from the start point, halfway through the path already, or
  driving full speed into it and the controller will handle it.
\end{quote}

Since you need odometry for the correction to work anyway, you know the starting
point. You could replan the trajectory instead, which can be cheap.
Alternatively, given a pose+velocity controller in the chassis coordinate frame,
you could adjust the ratio between the x and y position gains to change the
robot's approach.
\begin{quote}
  This also makes pathfinding a lot easier to work with. You can just take a
  list of points and follow them instead of having to worry about using them to
  create a continuous path, then generate all the path states with it.
\end{quote}

This goes back to my difference in philosophy: I want to control exactly what my
robot is doing at all times.
\begin{quote}
  Finally, tuning the controller doesn’t really seem necessary at the moment
  (there will still be the ability to do so), so getting it up and running is
  much simpler and faster.
\end{quote}

It does leave performance on the table though. In general, a feedback controller
arbitrarily deviating from the path would be considered poor controller tuning;
pure pursuit does that on purpose.

\section{Derivation}

$(x_r, y_r, \theta_r)$ is the robot's pose.

$(x_i, y_i)$ is the turning circle's intersection point with a path. It's the
closest point to robot that's on the path plus some lookahead distance along the
path.

$(x_c, y_c)$ is intersection circle's center. The robot's pose is tangent to the
circle.

We want to know the radius of the circle formed by the robot's pose tangent to
the circle and the intersection point. First, we'll find the center of the
circle by finding the intersection of the line through the robot's pose
perpendicular its heading, and the perpendicular bisector between $(x_i, y_i)$
and $(x_r, y_r)$. Then, we'll find the radius of this circle, which is the
turning radius of the robot.

We'll be using the point-slope form of a line $y - y_1 = m(x - x_1)$ throughout
this derivation.

\subsection{Line parallel to robot}

The line going through the robot's translation and parallel to the robot's
heading is
\begin{align*}
  y - y_1 &= m(x - x_1) \\
  y - y_r &= m(x - x_r) \\
  y - y_r &= \tan\theta_r (x - x_r) \\
  y - y_r &= \frac{\sin\theta_r}{\cos\theta_r} (x - x_r)
  \intertext{We want the line perpendicular to that so}
  y - y_r &= -\frac{\cos\theta_r}{\sin\theta_r} (x - x_r) \\
  \sin\theta_r (y - y_r) &= -\cos\theta_r (x - x_r) \\
  \sin\theta_r y - \sin\theta_r y_r &= -\cos\theta_r x + \cos\theta_r x_r \\
  \cos\theta_r x + \sin\theta_r y &= \cos\theta_r x_r + \sin\theta_r y_r
\end{align*}

\subsection{Perpendicular bisector between robot and intersection point}

We want the perpendicular bisector between $(x_i, y_i)$ and $(x_r, y_r)$.
The slope is $\frac{y_i - y_r}{x_i - x_r}$, and the slope of the perpendicular
bisector is thus $-\frac{x_i - x_r}{y_i - y_r}$. The bisector is located halfway
between $(x_i, y_i)$ and $(x_r, y_r)$.
\begin{align*}
  y - y_1 &= m(x - x_1) \\
  y - \frac{y_i + y_r}{2} &=
    -\frac{x_i - x_r}{y_i - y_r} \left(x - \frac{x_i + x_r}{2}\right) \\
  y - 0.5(y_i + y_r) &= -\frac{x_i - x_r}{y_i - y_r} (x - 0.5(x_i + x_r)) \\
  (y_i - y_r) (y - 0.5(y_i + y_r)) &= -(x_i - x_r) (x - 0.5(x_i + x_r)) \\
  (y_i - y_r) y - 0.5(y_i^2 - y_r^2) &= -(x_i - x_r) x + 0.5(x_i^2 - x_r^2) \\
  (x_i - x_r) x + (y_i - y_r) y &= 0.5(x_i^2 - x_r^2) + 0.5(y_i^2 - y_r^2) \\
  2(x_i - x_r) x + 2(y_i - y_r) y &= x_i^2 - x_r^2 + y_i^2 - y_r^2 \\
  2(x_i - x_r) x + 2(y_i - y_r) y &= (x_i^2 + y_i^2) - (x_r^2 + y_r^2)
\end{align*}

\subsection{Finding the circle center}

The circle center is where the two perpendicular lines intersect. Let's make a
system of linear equations to find it.
\begin{align*}
  \cos\theta_r x_c + \sin\theta_r y_c &= \cos\theta_r x_r + \sin\theta_r y_r \\
  2(x_i - x_r) x_c + 2(y_i - y_r) y_c &= (x_i^2 + y_i^2) - (x_r^2 + y_r^2)
  \intertext{}
  \begin{bmatrix}
    \cos\theta_r & \sin\theta_r \\
    2(x_i - x_r) & 2(y_i - y_r)
  \end{bmatrix}
  \begin{bmatrix}
    x_c \\
    y_c
  \end{bmatrix} &=
    \begin{bmatrix}
      \cos\theta_r x_r + \sin\theta_r y_r \\
      (x_i^2 + y_i^2) - (x_r^2 + y_r^2)
    \end{bmatrix}
\end{align*}

Now solve the system.
\begin{align*}
  \begin{bmatrix}
    x_c \\
    y_c
  \end{bmatrix} &=
    \begin{bmatrix}
      \cos\theta_r & \sin\theta_r \\
      2(x_i - x_r) & 2(y_i - y_r)
    \end{bmatrix}^{-1}
    \begin{bmatrix}
      \cos\theta_r x_r + \sin\theta_r y_r \\
      (x_i^2 + y_i^2) - (x_r^2 + y_r^2)
    \end{bmatrix} \\
  \begin{bmatrix}
    x_c \\
    y_c
  \end{bmatrix} &=
    \frac{1}{2(y_i - y_r)\cos\theta_r - 2(x_i - x_r)\sin\theta_r} \\
    &\qquad
    \begin{bmatrix}
      2(y_i - y_r) & -\sin\theta_r \\
      -2(x_i - x_r) & \cos\theta_r
    \end{bmatrix}
    \begin{bmatrix}
      \cos\theta_r x_r + \sin\theta_r y_r \\
      (x_i^2 + y_i^2) - (x_r^2 + y_r^2)
    \end{bmatrix}
\end{align*}

\subsection{Finding the turning radius}

The turning radius is the circle's radius.
\begin{align*}
  r = \sqrt{(x_r - x_c)^2 + (y_r - y_c)^2}
\end{align*}

\end{document}
